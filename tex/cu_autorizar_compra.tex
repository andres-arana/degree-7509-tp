\begin{tabularx}{\textwidth}{| r | X |}
\hline
\multicolumn{2}{|X|}{
\textbf{Use Case}: Autorizar compra} \\

\hline
\multicolumn{2}{|c|}{\cellcolor[gray]{0.6}} \\

\hline
\multicolumn{2}{|X|}{
\textbf{Descripción}: Validación de las condiciones de uso de una tarjeta para
realizar un pago en un comercio adherido.} \\

\hline
\multicolumn{2}{|X|}{
\textbf{Actores participantes}: Comercio} \\

\hline
\multicolumn{2}{|c|}{\cellcolor[gray]{0.6} } \\

\hline
\multicolumn{2}{|X|}{
\textbf{Flujos}} \\

\hline
\multicolumn{2}{|X|}{
\textbf{Flujo principal}} \\

\hline
1 & El comercio requiere al sistema que este valide una tarjeta informando el
número de esta, el DNI del cliente y el importe de la venta que el cliente
quiere abonar utilizando la misma. \\
\hline
2 & El sistema realiza las siguientes validaciones (S1 si alguna de las
validaciones fallase). 
\begin{enumerate}
\item El número de tarjeta debe estar asociado a una tarjeta.
\item La tarjeta debe estar activa.
\item El DNI del cliente asociado a la tarjeta debe coincidir con el DNI
informado.
\item El saldo de la cuenta asociada a la tarjeta más el importe informado de
la venta no debe superar el límite de saldo asociado a la cuenta.
\end{enumerate} \\
\hline
3 & El sistema informa al comercio que la compra está autorizada. \\
\hline
4 & Fin del caso de uso. \\

\hline
\multicolumn{2}{|X|}{
\textbf{Flujos alternativos}} \\

\hline
S1.1 & El sistema informa al comercio que la compra no está autorizada,
adjuntando en el mensaje el motivo por el cual no se autoriza la misma. \\
\hline
S1.2 & El caso de uso continua en 4. \\

\hline
\end{tabularx}

